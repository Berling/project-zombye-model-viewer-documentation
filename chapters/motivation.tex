\chapter{Motivation}
Bei der Entwicklung von 3D-Anwendungen ist es oft schwierig ein geeignetes Format zur Repräsentation der verwendenten Daten (etwa Geometrien, Animationen, usw.) zu finden, das alle gewünschten Informationen speichern kann. Im Zuge der Entwicklung einer 3D-Game-Engine kam eben dieses Problem auf. Dabei wurden viele der bekanntesten Formate untersucht und festgestellt, dass diese nicht die gewünschten Anforderungen erfüllen, zu komplex sind und sich damit nur unter erheblichem Aufwand in die Assetpipeline einbringen lassen oder die vorhandenen Tools und Programme unzureichend sind.

Diese Arbeit beschäftigt sich mit dem Entwurf eines solchen Formats und der Umsetzung eines Export Plugins, das alle Anforderungen erfüllt und genau auf den Anwendungsfall zugeschnitten ist.
