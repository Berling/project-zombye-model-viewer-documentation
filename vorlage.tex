%%%%%%%%%%%%%%%%%%% vorlage.tex %%%%%%%%%%%%%%%%%%%%%%%%%%%%%
%
% LaTeX-Vorlage zur Erstellung von Projekt-Dokumentationen
% im Fachbereich Informatik der Hochschule Trier
%
% Basis: Vorlage svmono des Springer Verlags
%
%%%%%%%%%%%%%%%%%%%%%%%%%%%%%%%%%%%%%%%%%%%%%%%%%%%%%%%%%%%%%

\documentclass[envcountsame,envcountchap, deutsch]{i-studis}

\usepackage{makeidx}         	% Index
\usepackage{multicol}        	% Zweispaltiger Index
%\usepackage[bottom]{footmisc}	% Erzeugung von Fu�noten

%%-----------------------------------------------------
%\newif\ifpdf
%\ifx\pdfoutput\undefined
%\pdffalse
%\else
%\pdfoutput=1
%\pdftrue
%\fi
%%--------------------------------------------------------
%\ifpdf
\usepackage[pdftex]{graphicx}
\usepackage[pdftex,plainpages=false]{hyperref}
%\else
%\usepackage{graphicx}
%\usepackage[plainpages=false]{hyperref}
%\fi

%%-----------------------------------------------------
\usepackage{color}				% Farbverwaltung
%\usepackage{ngerman} 			% Neue deutsche Rechtsschreibung
\usepackage[english, ngerman]{babel}
%\usepackage[latin1]{inputenc} 	% Erm�glicht Umlaute-Darstellung
\usepackage[utf8]{inputenc}  	% Erm�glicht Umlaute-Darstellung unter Linux (je nach verwendetem Format)

%-----------------------------------------------------
\usepackage{listings} 			% Code-Darstellung
\lstset
{
	basicstyle=\scriptsize, 	% print whole listing small
	keywordstyle=\color{blue}\bfseries,
								% underlined bold black keywords
	identifierstyle=, 			% nothing happens
	commentstyle=\color{red}, 	% white comments
	stringstyle=\ttfamily, 		% typewriter type for strings
	showstringspaces=false, 	% no special string spaces
	framexleftmargin=7mm,
	tabsize=3,
	showtabs=false,
	frame=single,
	rulesepcolor=\color{blue},
	numbers=left,
	linewidth=146mm,
	xleftmargin=8mm
}
\usepackage{textcomp} 			% Celsius-Darstellung
\usepackage{amssymb,amsfonts,amstext,amsmath}	% Mathematische Symbole
\usepackage[german, ruled, vlined]{algorithm2e}
\usepackage[a4paper]{geometry} % Andere Formatierung
\usepackage{bibgerm}
\usepackage{array}
\usepackage{amssymb}
\hyphenation{Ele-men-tar-ob-jek-te  ab-ge-tas-tet Aus-wer-tung House-holder-Matrix Le-ast-Squa-res-Al-go-ri-th-men} 		% Weitere Silbentrennung bei Bedarf angeben
\setlength{\textheight}{1.1\textheight}
\pagestyle{myheadings} 			% Erzeugt selbstdefinierte Kopfzeile
\makeindex 						% Index-Erstellung


%--------------------------------------------------------------------------
\begin{document}
%------------------------- Titelblatt -------------------------------------
\title{Blender 3D Model Export Plug-In}
%---- Die Art der Dokumentation kann hier ausgew�hlt werden---------------
%\project{Bachelor-Projektarbeit}
%\project{Bachelor-Abschlussarbeit}
%\project{Master-Projektstudium}
%\project{Master-Abschlussarbeit}
%\project{Seminar zur Vorlesung ...}
\project{Projektarbeit zur Vorlesung Tool- und Plugin-Programmierung}
%--------------------------------------------------------------------------
\supervisor{Prof. Dr. Christof Rezk-Salama} 		% Betreuer der Arbeit
\author{Georg Schäfer} 							% Autor der Arbeit
\address{Trier,} 							% Im Zusammenhang mit dem Datum wird hinter dem Ort ein Komma angegeben
\submitdate{\today} 				% Abgabedatum
%\begingroup
%  \renewcommand{\thepage}{title}
%  \mytitlepage
%  \newpage
%\endgroup
\begingroup
  \renewcommand{\thepage}{Titel}
  \mytitlepage
  \newpage
\endgroup
%--------------------------------------------------------------------------
\frontmatter
%--------------------------------------------------------------------------
\tableofcontents 						% Inhaltsverzeichnis
%--------------------------------------------------------------------------
\mainmatter                        		% Hauptteil (ab hier arab. Seitenzahlen)
%--------------------------------------------------------------------------
% Die Kapitel werden in separaten .tex-Dateien abgelegt und hier eingebunden.
\chapter{Motivation}
Bei der Entwicklung von 3D-Anwendungen ist es oft schwierig ein geeignetes Format zur Repräsentation der verwendenten Daten (etwa Geometrien, Animationen, usw.) zu finden, das alle gewünschten Informationen speichern kann. Im Zuge der Entwicklung einer 3D-Game-Engine kam eben dieses Problem auf. Dabei wurden viele der bekanntesten Formate untersucht und festgestellt, dass diese nicht die gewünschten Anforderungen erfüllen, zu komplex sind und sich damit nur unter erheblichem Aufwand in die Assetpipeline einbringen lassen oder die vorhandenen Tools und Programme unzureichend sind.

Diese Arbeit beschäftigt sich mit dem Entwurf eines solchen Formats und der Umsetzung eines Export Plugins, das alle Anforderungen erfüllt und genau auf den Anwendungsfall zugeschnitten ist.

\chapter{Aufbau des Formats}
Im Folgenden wird der Aufbau des \textit{Zombye Model Formats} (kurz zmdl) besprochen, welches für \textit{Project Zombye} Engine entwickelt wurde.

ZMDL ist ein Json-basiertes Modell-Format und wurde genau auf die Anforderungen der Engine zugeschnitten. Neben gewöhnlicher 3D-Geometrie werden auch Skelett-basierte Animationen unterstützt. Je nachdem, ob es sich bei dem Modell um statische Geometrie oder animierte Geometrie handelt, unterscheidet sich der Aufbau der einzelnen Datenstrukturen innerhalb des Formats.

\section{Aufbau statische Geometrie}
Das Format kann zwischen mehreren Objekten unterscheiden die über einen Namen mit ihren jeweiligen Daten asoziiert werden. Jedes Objekt besteht aus einer Liste von Vertices, einer Liste von Submeshes und einem Flag ob Parallax-Displacement-Mapping verwendet werden soll.

Die Vertices werden abgekapselt als eine Menge von Vertexattributen gespeichert. Dabei besteht jedes Vertex aus einer Position, einer Normalen und einer Texturkoordinate. Position und Normale sind aus jeweils drei-elmentigen Listen zusammengesetzt die Vektoren im $\mathbb{R}^3$ darstellen. Die Texturkoordinate hingegen ist eine zwei-elementige Liste, also ein Vektor im $\mathbb{R}^2$.

Die Submeshes sind eine Menge von Materialnamen die mit einem Submesh asoziiert werden. Ein Submesh ist eine Teilmenge der gesamten Indexmenge, gruppiert nach Material. Das bedeutet ein Submesh enthält alle Dreiecke die das gleiche Material teilen. Jedes Submesh besteht aus einer Liste an Indices, deren Elemente wiederum in Gruppen mit jeweils drei Elementen zusammengefasst sind (Triangles). Zudem enthält jedes Submesh drei Texturen und deren Pfad (\texttt{diffuse}, \texttt{material}, \texttt{normal}).

Im folgenden Beispiel wird ein einziges Dreieck im ZMDL-Format dargestellt.
\vspace{3cm}

\begin{lstlisting}
    {
        "triangle" : {
            "vertices" : [
                {
                    "position" : [-0.5, -0.5, 0.0],
                    "texcoord" : [0.0, 0.0],
                    "normal" : [0.0, 1.0, 0.0]
                },
                {
                    "position" : [0.5, -0.5, 0.0],
                    "texcoord" : [1.0, 0.0],
                    "normal" : [0.0, 1.0, 0.0]
                },
                {
                    "position" : [0.0, 0.5, 0.0],
                    "texcoord" : [0.5, 0.5],
                    "normal" : [0.0, 1.0, 0.0]
                }
            ],
            "submeshes" : {
                "material" : {
                    "indicies" : [
                        [0, 1, 2]
                    ],
                    "textures" : {
                        "diffuse" : "path/diffuse.dds",
                        "material" : "path/material.dds",
                        "normal" : "path/normal.dds"
                    }
                }
            },
            "parallax" : false
        }
    }
\end{lstlisting}

\section{Aufbau animierter Geometrie}
Alle bisher genannten Daten sind wie beschrieben auch in animierter Geometrie vorhanden, werden allerdings teilweise erweitert.
So gibt es zwei neue Vertex-Attribute, \texttt{indices} und \texttt{weights}, die in jedem Vertex vorhanden sind. Die Indices sind die der Bones die das Vertex beeinflussen. Die Weights geben an wie stark der Bone den Vertex transformiert. Jedem Vertex können bis zu vier Bones gleichzeitig zugewiesen werden.

Jedes animierte Objekt besitzt zwei zusätzliche Datensätze. Zum Einen die Bonehierchie, die aus einer Zuordnung zwischen Parent- und Child-Bone besteht. Ein Parent kann mehrere oder keine Children haben. Zum Anderen die Liste der Animationen. Diese sind einem Namen zugeordnet und bestehen aus einer Liste an sog. Tracks. Diese repräsentieren jeweils einen Bone. Jeder Track enthält sämtliche Keyframes für diesen Bone innerhalb der Animation. Die Keyframes sind unterteilt nach Translation, Rotation und Skalierung.

\chapter{Implementierung des Export-Plugins}
Das Export-Plugin wurde für Blender geschrieben. Plugins in Blender werden vorrangig in Python geschrieben und als Python Module realisiert. Die Grundlegende Herangehensweise bei der Implementierung war zunächst API-Dokumentation nach den benötigten Informationen durchsuchen, die Daten in Python anfordern, alle Daten in einem großen Dictionary sammeln und diese anschließend in eine Datei schreiben.

Dafür bot sich JSON als Basis an, da Python bereits einen JSON-Parser bzw. Dumper mit sich bringt und alle Datenstrukturen aus JSON auf Äquivalente in Python umgesetzt werden können. Schon zu Begin der Entwicklung wurde festgestellt, dass die Dokumentation der Blender API sehr unvolständig und teilweise auch veraltet ist \footnote{An einigen Stellen in der Dokumentation werden Funktionen mit \textit{blabla} dokumentiert}. Wenig hilfreich war auch die Inkonsistenz der API. Oft war Try and Error der bevorzugte Ansatz, da weder die Dokumentation noch einschlägige Beiträge in Foren Abhilfe schafften.

In der Retrospektive war der Aufwand im Verhältnis zum Nutzen des Plugins sehr hoch. Die aktuelle Version des Plugins unterstützt keine IK-Solver oder ähnliches. Es können also nur die Keyframes des tatsächlichen Skelets exportiert werden.

\chapter{Bedienung}
\section{Plugin Installieren}
Das Plugin wurde mit der Blender Version 2.75a getestet. Folgende Schritte sind zur Installation des Plugins notwendig:

\begin{enumerate}
    \item Blender starten
    \item File $\rightarrow$ User-Preferences $\rightarrow$ Add-ons $\rightarrow$ Install from File
    \item project-zombye-exporter.zip im Projekt Ordner auswählen
    \item Install from File
    \item Haken setzen (Import-Export: Zombye Model Exporter)
    \item Save User Settings
\end{enumerate}

Nach der Installation befindet sich das Plugin unter File $\rightarrow$ Export $\rightarrow$ Zombye Model (.zmdl). Im Ordner \texttt{example-models} befinden sich zwei blend-Dateien, ein statisches Objekt und ein animiertes Objekt, die mit dem Plugin exportiert werden können.

Das Plugin bietet die Möglichkeit nur die ausgewählten Objekte in einer Scene zu exportieren. Dazu muss die Option \texttt{selected} gesetzt werden.

\section{Kompilierung}
Dem Plugin liegt auch noch ein Model-Viewer bei, mit dessen Hilfe die exportierten Dateien betrachtet werden können.

Die Kompilierung unter Windows wurde mit dem in der MinGW Distribution TDM (Version 5.1) enthaltenen GCC getestet. Folgender Schritte sind zur Kompilierung nötig. Es wird angenommen, dass der Pfad zu Premake in \texttt{PATH} eingetragen ist. Sollte dies nicht der fall sein, muss der vollständige Pfad zu Premake beim aufrufen angegeben werden.

\begin{enumerate}
    \item MinGW Konsole öfnnen
    \item Ins Verzeichnis \texttt{project-zombye-model-viewer} wechseln
    \item premake5 gmake
    \item mingw32-make.exe -j 4 CC=gcc CXX=g++
\end{enumerate}

Vor dem Start müssen die ZMDL-Dateien noch in das binäre Format der Project-Zombye-Engine gebracht werden. Die geschieht mit dem Meshconverter bzw. Animationconverter.

\begin{enumerate}
    \item mesh\_converter\textbackslash{}mesh\_converter.exe path\textbackslash{}human.zmdl assets/
    \item animation\_converter\textbackslash{}animation\_converter.exe path\textbackslash{}human.zmdl assets/
    \item build\textbackslash{}zmdl\_viewer.exe assets\textbackslash{}meshes\textbackslash{}human.msh assets\textbackslash{}anims\textbackslash{}human.skl
\end{enumerate}

\begin{enumerate}
    \item mesh\_converter\textbackslash{}mesh\_converter.exe path\textbackslash{}weighted\_companion\_cube\_export.zmdl assets/
    \item build\textbackslash{}zmdl\_viewer.exe assets\textbackslash{}meshes\textbackslash{}cube.msh
\end{enumerate}

Der Viewer verwendet die Arc-Ball Klasse aus der Tool- und Pluginprogrammierungs Übung. Die Steuerung der Kamera ist genau so über die Maus steuerbar. Wird im Viewer ein animiertes Model geladen, kann zwischen den Animationen mit Rechts bzw. Links hin und her gewechselt werden. Das Starten und Stoppen der Aimation ist mit der Leer-Taste möglich.

% ...
%--------------------------------------------------------------------------
\backmatter                        		% Anhang
%-------------------------------------------------------------------------
\bibliographystyle{geralpha}			% Literaturverzeichnis
\bibliography{literatur}     			% BibTeX-File literatur.bib
%--------------------------------------------------------------------------
%--------------------------------------------------------------------------
\begin{appendix}						% Anh�nge sind i.d.R. optional
\end{appendix}

\end{document}
